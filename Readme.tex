\documentclass[10pt,a4paper]{article}
\usepackage[utf8]{inputenc}
\usepackage{amsmath}
\usepackage{amsfonts}
\usepackage{amssymb}
\usepackage{fullpage}
\begin{document}
\begin{center}
\Large\textbf{Long Term Time Series Generation Conditioned on Sequential Short-Term Forecasts} \\
\large{Kyle Perline}
\end{center}



This code implements the Long Term Generation algorithm.  This algorithm was developed in Kyle Perline's Ph.D. thesis at Cornell University, and this work is currently under submission to IEEE Transactions on Sustainable Energy.

The Long Term Generation algorithm generates long term time series conditioned on sequential short-term historical forecasts. The application is to generating wind power scenarios conditioned on sequential short-term historical wind power forecasts.

\section{Technical Definitions and Objective}

Let the time index be $t$.  Let $R_t$ be a (univariate) response random variable. Let $F_t=(F_t^1,...,F_t^{\Delta})$ be a $\Delta$-dimensional random vector of forecasts, where $F_t^i$ predicts the response variable $R_{t+i}$, for each $i=1,...,\Delta$. (The predictor variables $F_t$ can be more general than the described forecasts; e.g. they can be probabilistic forecasts instead of point forecasts, or they do not even need to be forecasts at all.) Over time steps $T_1,...,T_2$ there is some joint probability density function $P$ of the $R_t$ and $F_t$, denoted

$$ P(F_{T_1},R_{T_1},F_{T_1+1},R_{T_1+1},...,F_{T_2},R_{T_2}). $$

\noindent Let $r_t$ and $f_t$ be historical samples of the random variables $R_t$ and $F_t$, respectively.

Suppose that historical forecasts $f_{T_1},...,f_{T_2}$ are obtained.  Then the conditional distribution of the response variables conditioned on the forecasts is

$$ P(R_{T_1},...,R_{T_2} | f_{T_1},...,f_{T_2}). $$

The Long Term Generation algorithm creates an estimate $\hat{P}$ of this conditional distribution, i.e.

$$ \hat{P}(R_{T_1},...,R_{T_2} | f_{T_1},...,f_{T_2}) =_d P(R_{T_1},...,R_{T_2} | f_{T_1},...,f_{T_2}). $$


\noindent This Long Term Generation algorithm has three main steps:

\begin{enumerate}
\item For each time step $t=T_1,...,T_2$ estimate the \textbf{marginal distribution} $P(R_t|f_{T_1},...,f_{T_2})$.

This is accomplished by:
\begin{enumerate}
\item Create a predictor random variable $X_{t,u}=h_{t,u}(F_{T_1+u},...,F_{T_2+u})$; let $x_{t,u}$ be the historical sample.
\item For some positive integer, draw historical samples 
$$D_t = \{ (x_{t,u},r_t) \}_{-N\leq u \leq N}.$$
This definition assumes that $P$ is \textit{slowly time varying}.
\item Use a numerical method to construct the estimate $\hat{P}(R_t|X_{t,0})$ based on $D_t$.  The two numerical methods that have been implemented are \textit{Kernel Density Estimation} and \textit{Quantile Regression}.
\end{enumerate}
For example, if $X_{t,u}=F_{t-1+u}^1$, then the predictor variable is the previous time step's one time step ahead forecast.  Then in step (b) the data set at time $t$ is
$$ D_t = \{ (f_{t-1+u}^1,r_{t+u}) \}_{-N \leq u \leq N}. $$
\noindent In step (c) the slowly time varying assumption means that each data point in $D_t$ is drawn from the same distribution $P(R_t|F_{t-1}^1)$.  We therefore construct the estimate $\hat{P}(R_t|F_{t-1}^1)$.

\item Construct the \textbf{joint distribution} $\hat{P}(R_{T_1},...,R_{T_2}|X_{T_1,0},...,X_{T_2,0})$ based on the marginal distributions.  A Gaussian copula approach is used, where each of the marginal distributions are transformed into a standard normal distribution.  The joint distribution is then uniquely specified by the covariance matrix.  There are multiple methods for estimating or constructing this covariance matrix.
\item Draw scenarios from $\hat{P}(R_{T_1},...,R_{T_2}|X_{T_1,0},...,X_{T_2,0})$.
\end{enumerate}

\section{Long Term Generation code}
There are three main sections of code.

\begin{enumerate}
\item Marginal Distributions:  Both Kernel Density Estimation (KDE) and Quantile Regression (QR) have been implemented.  Run \texttt{KDEdemo} for an example of KDE usage.
\item Joint Distribution: The Gaussian copula method has been implemented, with various approaches for estimating the covariance matrix.
\item Getting Data: Various methods were implemented for automatically obtaining the data set $D_t$ in step 1b.
\end{enumerate}

Wind scenarios can be generated using the Long Term Generation algorithm by calling \texttt{RUN\_GEN\_SCENARIOS.LTG()}.

\end{document}